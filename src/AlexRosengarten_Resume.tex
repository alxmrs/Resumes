% LaTeX resume using res.cls
\documentclass[line,mm]{res}

\usepackage[letterpaper, top=.10in, bottom=.10in, left=.25in, right=.25in, marginparwidth=.01in]{geometry}

\usepackage{hyperref}
\hypersetup{
    colorlinks=true,
    linkcolor=blue,
    urlcolor=blue
}

\begin{document}

    \moveleft\hoffset\centerline{\large\textbf{Alexander Sky Rosengarten} }
    % address begins here
    % Again, the address lines must be centered over entire width of resume:
    \moveleft\hoffset\centerline{alxrsngrtn@google.com | \href{https://github.com/alxrsngrtn/}{github.com/alxrsngrtn} | \href{https://alexrosengarten.com}{alexrosengarten.com} | Oakland, CA }

    % Draw a horizontal line the whole width of resume:
    \moveleft\hoffset\vbox{\hrule width 7.5in height 1pt}\smallskip

    % ---------------- EDUCATION ----------------

    \section{EDUCATION}\label{sec:education}
    \hspace*{\fill}
    {\textsl{B.S.} } Computer Science \& {\textsl B.S.} Cognitive Science \hfill University of California, San Diego \hfill Class of 2016
    \hspace*{\fill}

    % ---------------- TECHNICAL SKILLS ----------------

    \section{TECHNICAL SKILLS}\label{sec:technical-skills}
    \begin{tabular}{l@{\hskip 0.25in} l}
        \textbf{Languages} & Kotlin, Typescript, Python, Java, C++\\
        \textbf{Technologies} & Tensorflow, Android, PyData Stack, Reactive Extensions, OpenCV, AWS \\
        \textbf{Environment} & Git, GitHub, Bazel, Bash, Vim, Docker, TravisCI
    \end{tabular}

    % ---------------- WORK EXPERIENCE ----------------
    % Talk about the challenge that had to be done, the  actions that you took to tackle the challenge, and the results (how things outside of yourself were experienced)
    \section{WORK EXPERIENCE}\label{sec:work-experience}
    {\textbf Software Engineer in Machine Learning }  2017/11 | 2018/03 \\
    Aira Tech Cop -- La Jolla \& San Jose, CA \\
    \textit{ Aira helps blind and low-vision users access visual information via remote assistance with smart glasses. }
    \begin{itemize}
        \itemsep -2pt
        \item Built {\textbf{core dialog engine}  (Chloe)} on Android. Integrated speech/text interface and conversation state manager. \\
              Built features such as bluetooth pairing, calling an agent, and call rating purely through a voice interface (Java 8).
        \item Developed cloud-based, \href{https://github.com/aira/object_detector}{{\textbf real-time object detector}}. AI agent used NLP, computer vision, and deep learning for \\
              NSF grant, designed to assist blind \& low-vision users (Tensorflow, OpenCV, MQTT).
        \item Technical lead for prototype of {\textbf indoor navigation system} (OpenSfM, PCL, Python, Android, ArchGIS).
        \item Project lead for {\textbf Horus}, a batch ETL \& analytics tool for video streams (Python, Docker, AWS).
        \item Created {\textbf image tagging game} to generate dataset integral to company IP (Spring Boot, Vue.js, Typescript).
        \item Created {\textbf currency classifier}, a CNN model. Deployed on Android with Firebase (Tensorflow, Android).
        \item Added {\textbf fisheye correction} in glasses camera driver (OpenCV, Android Native).
        \item Lead agile rituals such as daily standups, sprint plannings and retrospectives.
    \end{itemize}

    {\textbf Research Assistant } \hfill 2018/01 | 2018/04 \\
    Vecchio Group, UCSD Nanoengineering Department -- La Jolla, CA
    \begin{itemize}
        \itemsep -2pt
        \item Developed interpretable CNN model to automate crystallography detection, leading to publication in \href{https://science.sciencemag.org/content/367/6477/564.abstract}{\it Science}.
        \item Used Random Forests algorithm to design non-trivial, novel ceramics. Material designs passed experimental verification.
        \item Used OOP design principles for framework to track hyperparameters, ML pipelines in version control (Python).
        \item {\textbf Teaching graduate students} programming best-practices and machine learning concepts.
    \end{itemize}

    {\textbf Software Engineer} \hfill 2016/08 | 2017/11 \\
    Intuit -- San Diego, CA
    \begin{itemize}
        \itemsep -2pt
        \item Developed features for {\textbf ProSeries}, Windows professional tax software (C\#, C++).
        \item {\textbf Won business unit hackathon} with novel idea for a telemetry system, presented to senior leadership.
        \item {\textbf Led telemetry project}. Took initiative to modernize legacy application with data.
        \item Implemented {\textbf application crash analysis tool}, produced more accurate reports in half the time (Python).
        \item Ideas I designed and executed led to a 50\% reduction in application crashes from prior year.
    \end{itemize}


    % ---------------- INTERESTS ----------------

    \section{INTERESTS}\label{sec:interests}
    Artificial Intelligence, Biometrics, Functional Programming, Teaching, Accessibility

    % ---------------- ACTIVIES & AWARDS ----------------
    %\section{ AWARDS \& ACTIVITIES }\label{sec:awards-and-activies}
    %  {\textbf Third Place Winner}, {\it IEEE Brain Computer Interface Hackathon} \hfill September 2016\\
    %  {\textbf Computer Science Tutor}, {\it UCSD CSE Department} \hfill Spring 2013, Summer 2014 \\
    %  {\textbf Summer Research Scholar}, {\it Qualcomm Institute/Calit2} \hfill June - August 2013 \\
    %  {\textbf President}, {\it Cognitive Science Student Association} \hfill August 2012 - January 2014 \\

\end{document}

