% LaTeX resume using res.cls
\documentclass[line,margin]{res} 
%\usepackage{helvetica} % uses helvetica postscript font (download helvetica.sty)
%\usepackage{newcent}   % uses new century schoolbook postscript font 


%\topmargin=-.10in
%\oddsidemargin -.5in
%\evensidemargin -.5in
%\textwidth=6.0in
%\itemsep=0in
%\parsep=0in

\usepackage[letterpaper, tmargin=.10in, bmargin=.10in, lmargin = .25in, rmargin=1.75in, marginparwidth=.01in]{geometry}
\usepackage{hyperref} 


\begin{document}

% Center the name over the entire width of resume:
 \moveleft.5\hoffset\centerline{\large\bf Alexander Sky Rosengarten}
% address begins here
% Again, the address lines must be centered over entire width of resume:
 \moveleft.5\hoffset\centerline{alex.rosengarten@gmail.com | \href{https://github.com/alxrsngrtn/}{github.com/alxrsngrtn} | (760) 518-7440 | San Diego, CA }

% Draw a horizontal line the whole width of resume:
 \moveleft\hoffset\vbox{\hrule width 7.5in height 1pt}\smallskip
 
%\section{OBJECTIVE}
%Looking to make an meaningful impact as a machine learning engineer or full-stack web developer. 
 
\section{EDUCATION}  {\sl B.S.} Computer Science \& {\sl B.S.} Cognitive Science \hfill UC San Diego --  La Jolla, CA  \hfill Class of 2016 

\section{TECHNICAL \\ SKILLS} 
\begin{tabular}{l@{\hskip 0.25in} l}
\textbf{Languages} & Python, Java, C++, C\#, JavaScript, Scala/F\#/OCaml \\
\textbf{Technologies} & Android, Reactive Extensions, SQL/No-SQL, AMQP, REST, AWS, Google Cloud \\
\textbf{Data Science} & Keras/Tenserflow, Scikit-Learn, OpenCV, Numpy, Pandas, Jupyter, Apache Spark \\
\textbf{Environment} & POSIX, Vim, JetBrains IDEs, Git, Docker, Vagrant, Jira/Confluence, Windows
\end{tabular}

% Talk about the challenge that had to be done, the  actions that you took to tackle the challenge, and the results (how things outside of yourself were experienced)		
\section{WORK EXPERIENCE} 
{\bf Machine Learning/Software Engineer } \hfill Nov 2017 - Present \\
  Aira -- La Jolla, CA
  \begin{itemize}  \itemsep -2pt %reduce space between items
    \item Developing {\bf Chloe}, a mobile-AI for new product line (Android, RxJava). 
    \item Building {\bf core dialog engine}. Includes a STT/TTS system and conversation state management. Built features such as bluetooth pairing, calling an agent, and call rating purely throuch a voice interface. 
    \item Built cloud-based, \href{https://github.com/aira/object_detector}{{\bf real-time object detector}}.  AI agent used NLP, computer vision, and deep learning for NSF grant, designed to assist blind/low-vision users. (Tensorflow, OpenCV, MQTT).
    \item AI agent can describe the color, amount, and type of 100 classes of objects in milliseconds.
    \item Created {\bf wake word detector}. Built a CNN model and audio processing pipeline (Tensorflow).
    \item Built a {\bf telemetry system} that captures speech utterances from users and the device. Speech simulation added for automated testing (AWS IoT, MQTT, DynamoDB).
    \item Added {\bf fisheye correction} in glasses camera driver (OpenCV, Android Native).
    \item Started a weekly lecture series to teach engineers machine learning. 
  \end{itemize} 


{\bf Software Engineer} \hfill Aug 2016 - Nov 2017 \\
  Intuit -- San Diego, CA
  \begin{itemize}  \itemsep -2pt %reduce space between items
    \item Developed features for {\bf ProSeries}, Windows professional tax software (C\#, C++).
    \item {\bf Won business unit hackathon} with novel idea for a telemetry system, presented to senior leadership.
    \item {\bf Led telemetry project}, which was an outstanding need for over three years. Telemetry system was critical for new design/refactoring initiatives and informing business decisons.
    \item Wrote \href{https://github.com/alxrsngrtn/CrashAnalysisTool}{{\bf application crash analysis tool}}, produced more accurate reports in half the time (Python).
    \item Started weekly, technical engineering meeting. Taught functional programming, facilitated group discussions about Code Complete and design patterns, improved agile/development process.
  \end{itemize} 


{\bf Software Engineering Intern} \hfill            Mar - Aug 2016 \\
  Ingenu -- San Diego, CA
  \begin{itemize}  \itemsep -2pt %reduce space between items
    \item Developed {\bf Supernova}, a single-page web application to manage IoT devices (Angular 1).
    \item�Reduced application load time by 44\% and KB size by 40\% via minification, lazy loading, and CDN.
    \item Prototyped, designed, and developed {\bf RSS feed widget} (Spring boot, Angular).
  \end{itemize} 
	
{\bf Research Assistant} \hfill            Jun 2013 - Sep 2015 \\
   Natural Computation Laboratory, UCSD Cognitive Science Department -- La Jolla, CA
   \begin{itemize}  \itemsep -2pt %reduce space between items
     \item Created \href{https://youtu.be/UMACp0fc9TA}{{\bf in-ear EEG}} device to detect seizures for people with epilepsy (SVMs, OpenBCI).
     \item Built custom neural network layer for \href{https://github.com/alxrsngrtn/LearnedNormPooling}{{\bf learned-norm pooling}} in EEG research application (Theano).
     \item Created hands-on workshops to {\bf teach brain computer interface technology} (OpenBCI, Python).
     \item Developed \href{https://github.com/alxrsngrtn/BrainTag}{{\bf BrainTag}}: a lasertag-based neurofeedback game for children with Autism (Arduino, C). 
     \item Awarded Summer Research Scholar at the Qualcomm Institute/Calit2, presented research at Undergraduate Research Conference.
    %\item Wrote data visualization and machine learning toolbox for open source brain-computer interface system (Python, Java, C, Javascript). 
   \end{itemize} 
	
    
\section{PROJECTS}
{\bf Research Assistant } \hfill            Jan 2018 - Present \\
  Vecchio Group, UCSD Nanoengineering Department -- La Jolla, CA
  \begin{itemize}  \itemsep -2pt %reduce space between items
    \item Building {\bf crystal structure classifier}. Built a data pipeline and model to infer crystal structure types based on images of defraction patterns, achieved 94\% accuracy across four classes (Keras/Tensorflow).
    \item {\bf Teaching graduate students} programming best-practices and machine learning concepts.  
  \end{itemize} 


{\bf Huh, Computer Music (HCM) } \hfill 	 Oct 2017 - Present 
  \begin{itemize}  \itemsep -2pt % reduce space between items
    \item Developing \href{https://github.com/alxrsngrtn/huh-computer-music}{{\bf open-source}} Python 3 library for real-time {\bf music generation}.  
    \item Presented {\bf ``\href{https://youtu.be/bTAFl8P2DkE?t=28m47s}{Music from Chaos: Audiofying the Lorenz Attractor}''} at PyCon, 2018.
    \item Library utilizes functional reactive programming for live streaming (RxPy).
    \item Tool aims to mimic an analog synthesizer through functional composition (Numpy, Scipy).
  \end{itemize}

\section{INTERESTS}
Artificial Intelligence, Biometrics, Functional Programming, Category Theory, Teaching, Accessibility
                 
%\section{ AWARDS \& ACTIVITIES }  
%  {\bf Third Place Winner}, {\it IEEE Brain Computer Interface Hackathon} \hfill September 2016\\
%  {\bf Computer Science Tutor}, {\it UCSD CSE Department} \hfill Spring 2013, Summer 2014 \\
%  {\bf Summer Research Scholar}, {\it Qualcomm Institute/Calit2} \hfill June - August 2013 \\   
%  {\bf President}, {\it Cognitive Science Student Association} \hfill August 2012 - January 2014 \\
            
\end{document}







