% LaTeX resume using res.cls
\documentclass[line,margin]{res} 
%\usepackage{helvetica} % uses helvetica postscript font (download helvetica.sty)
%\usepackage{newcent}   % uses new century schoolbook postscript font 


%\topmargin=-.10in
%\oddsidemargin -.5in
%\evensidemargin -.5in
%\textwidth=6.0in
%\itemsep=0in
%\parsep=0in

\usepackage[letterpaper, tmargin=.10in, bmargin=.10in, lmargin = .25in, rmargin=1.75in, marginparwidth=.01in]{geometry}



\begin{document}

% Center the name over the entire width of resume:
 \moveleft.5\hoffset\centerline{\large\bf Alexander S. Rosengarten}
% Draw a horizontal line the whole width of resume:

 \moveleft\hoffset\vbox{\hrule width 7.5in height 1pt}\smallskip
% address begins here
% Again, the address lines must be centered over entire width of resume:
 \moveleft.5\hoffset\centerline{alex.rosengarten@aira.io | 5328 Middleton Road, San Diego, CA, 92109 | github.com/alxrsngrtn}
 
%\section{OBJECTIVE}
%Looking to make an meaningful impact as a machine learning engineer or full-stack web developer. 
 
\section{EDUCATION} University of California, San Diego, {\it La Jolla, CA} \hfill {\sl B.S.} Computer Science \& {\sl B.S.} Cognitive Science
		  

\section{TECHNICAL \\ SKILLS} 
	
%\begin{tabular}{l l l}
% 								Experience & Languages & Technologies \\ \hline
%								Adept	&   \parbox[t]{2.4in}{Python, Java, Javascript, C/C++,  Matlab/Julia}   & %\parbox[t]{2.4in}{Numpy/Pandas, Angular.js, Webpack, Git, Theano, Jet Brains IDEs} \\
%								Proficient &  \parbox[t]{2.4in}{C\#, Scala} & \parbox[t]{2.4in}{Docker, Express.js, Ionic/%Cordova, SQL, Apache Spark, Spring MVC, }\\ % Hadoop
%								% Adobe Creative Suite, Android Studio, Arduino, Eclipse, GDB, Git, Unix, Vim
%\end{tabular}


%Languages & Python, C\#, C++, JavaScript, Java, Matlab/Octave/Julia, F\#, Scala  \\
%Frontend & Angular.js, Jasmine, Ionic/Cordova, Grunt/Gulp, Webpack \\
%Backend & SQL, Cassandra, Flask, Node.js, Django, Neo4j, Spring boot \\
%Data & Keras/Tenserflow/Theano, Scikit-Learn, Apache Spark \\
%Environment & Git, Docker, Vagrant, Unix/Windows, SVM, Perforce
%Frontend & Android, Angular.js, HTML5/CSS3, Webpack \\
%Backend & SQL, Cassandra, MQTT, REST, Flask, Django, Express.js, AWS \\
%Data & Keras/Tenserflow, OpenCV, Scikit-Learn, Pandas, Apache Spark  \\

\begin{tabular}{l l}
Languages & Python, Java, C++, C\#, JavaScript, Matlab, Scala/F\# \\
Technologies & Android, Reactive Extensions, Angular.js, HTML5/CSS3, SQL/No-SQL, MQTT, Keras/Tensorflow, OpenCV, Scikit-Learn, Apache Spark
Environment & POSIX/Windows, Docker, Vagrant, Jenkins, Git, SVM
\end{tabular}


% Talk about the challenge that had to be done, the  actions that you took to tackle the challenge, and the results (how things outside of yourself were experienced)		
\section{WORK EXPERIENCE} 
{\sl Machine Learning/Fullstack Engineer } \hfill            December 2017 - Present \\
                Aira, La Jolla, CA
                 \begin{itemize}  \itemsep -2pt %reduce space between items
                \item Built realtime, cloud-based AI Agent using NLP, computer vision, and deep learning for NSF grant. Tool designed to assist blind and low-vision users. (Tensorflow, OpenCV, MQTT, Python 3).
                \item AI Agent can describe the color, amount, and type of 100 classes of objects in under half a second.
		\item Developing a mobile-AI named Chloe for new product line (Android, RxJava, Java 8). 
		\item Building core dialog engine. Includes a speech-to-text/text-to-speech system, intent matching, and conversation state management. 
                \item Built voice telemetry system, captures speech utterances from users and the device.Has speech simulation for automated testing (AWS IoT, MQTT, DynamoDB).
                \item Implemented fisheye correction in the camera driver later (Android Native, OpenCV, C++). 
                \item Created wake word detector and audio processing pipeline (Tensorflow, Android, Python 3)
                 \end{itemize} 

{\sl Research Assistant } \hfill            January 2018 - Present \\
                UCSD Nanoengineering Department, La Jolla, CA
                 \begin{itemize}  \itemsep -2pt %reduce space between items
                \item Investigating the applications of deep learning in
                 \end{itemize} 

		{\sl Software Engineer} \hfill            August 2016 - November 2017 \\
                Intuit, San Diego, CA
                 \begin{itemize}  \itemsep -2pt %reduce space between items
                 \item Developed C\# and C++ features for ProSeries, Windows professional tax software -- a 1.2+ million line codebase. 
                 \item Lead telemetry project, filling an outstanding need of over three years. Project was key to instrumenting new designs and refactoring initiatives. 
                 \item Won business unit hackathon with novel idea for a telemetry system, presented to senior leadership.
                 \item Wrote application crash analysis tool that produces more accurate reports in half the time (Python 3).
                 \item Managed software-focused book club.
                 \end{itemize} 


%{\sl Content Developer } \hfill            February - April 2016 \\
%                AI-Master, San Diego, CA
%                 \begin{itemize}  \itemsep -2pt %reduce space between items
%                \item Wrote interactive tutorials in Python at Machine Learning education start-up.
%                \item Developed lesson plans on Support Vector Machines in Jupyter/IPython Notebooks.
%                \item Established version control system (Github) for content team using open source pull-request model.
%                 \end{itemize} 
%

		{\sl Software Engineering Intern} \hfill            March - August 2016 \\
                Ingenu, San Diego, CA
                 \begin{itemize}  \itemsep -2pt %reduce space between items
                 \item Developed single-page web application with Angular 1 framework.
                 \item�Reduced application load time by 44\% and KB size by 40\% via minification, lazy loading, and CDN.
                 \item Prototyped, designed, and developed RSS feed sanitizer and aggregator (Spring boot, Angular).
                 \end{itemize} 
	
		{\sl Research Assistant} \hfill            June 2013 - September 2015 \\
                Natural Computation Laboratory, UCSD Dept. of Cognitive Science, La Jolla, CA
                 \begin{itemize}  \itemsep -2pt %reduce space between items
                 \item Build custom CNN layer for learned norm pooling in EEG deep learning application (Theano).
                 \item Created hands-on workshops to teach brain computer interface technology (OpenBCI, Python).
                 \item Developed BrainTag: open source neurofeedback game for children with Autism (Arduino, C). 
                 
                 %\item Wrote data visualization and machine learning toolbox for open source brain-computer interface system (Python, Java, C, Javascript). 
                 \end{itemize} 
	
		%{\sl Computer Science Tutor} \hfill 	April - June 2013, August 2014 \\
		%UCSD Computer Science Department, La Jolla, CA
                 %\begin{itemize}  \itemsep -2pt % reduce space between 
                 %\item Spring 2013, Matlab Programming: Tutored students, debugged code, graded exams and assignments.
                 %\item Summer 2014, Systems Programing: Tutored students in C and ARM Assembly programming course, implemented on Raspberry Pi. Found errors and debugged code (ARM, C, Raspberry Pi). 
                 %\end{itemize}
                 
    
\section{PROJECTS}
		 {\sl Huh Computer Music (HCM)} \hfill 	 October 2017 - Present 
                 \begin{itemize}  \itemsep -2pt % reduce space between items
                 \item Developing Open-Source Python 3 library for realtime music generation.  
                 \item Library utilizes functional reactive programming for live streaming (RxPy).
                 \item Tool aims to mimic an analog synthesizer through functional composition (Numpy, Scipy).
                 \end{itemize}
                 % include quantitative results
%                {\sl Forward 135 - Full Stack Developer} \hfill 	 April - June 2016 \\
%                Server-Side Web Applications Course, La Jolla, CA 
%                 \begin{itemize}  \itemsep -2pt % reduce space between items
%                 \item Built a full-stack architecture for efficient change propagation from server to client (JS ES6, Webpack)
%                 \item Implemented MVVM solution that sends only diffs from database to client via WebSockets (Node.js). 
%                 \item Performed Incremental View Maintenance via materialized views and database triggers (PostgreSQL). 
%                 \end{itemize}
                 
		 {\sl In-Ear EEG - Software Team Member} \hfill 	 April - June 2015 \\
                Embedded Systems Project Course, La Jolla, CA 
                 \begin{itemize}  \itemsep -2pt % reduce space between items
                 \item Worked in team to create In-Ear EEG device for people with epilepsy.
                 \item Used ensemble of SVMs to detect presence of epileptic seizures in realtime data (Matlab, OpenBCI).
                 \item Prototyped server that stored 20 minutes of brainwave data once seizures are detected (Python, SQL). 
                 \end{itemize}
                
\section{COURSEWORK}
   
   \begin{tabular}{l l l}
   $\bullet$ Advanced Data Structures  & $\bullet$ Advanced Machine Learning & $\bullet$ Algorithms \\
   $\bullet$ Compiler Construction & $\bullet$ Computer Vision & $\bullet$ Neural Networks \\ 
   $\bullet$ Operating Systems  &  $\bullet$ Programming Languages & $\bullet$ Server-Side Web Applications  \\
     
    \end{tabular}
  
                 
\section{ AWARDS \& ACTIVITIES }  
	    Third Place Winner, {\it IEEE Brain Computer Interface Hackathon} \hfill September 2016\\
	    Trainee, {\it Temporal Dynamics of Learning Center} \hfill January 2014 - June 2015 \\  
   	    Founding Member, {\it Data Science Student Society} \hfill November 2014 - September 2015 \\
	    Computer Science Tutor, {\it UCSD CSE Department} \hfill Spring 2013, Summer 2014 \\
	    Summer Research Scholar, {\it Qualcomm Institute/Calit2} \hfill June - August 2013 \\   
            President, {\it Cognitive Science Student Association} \hfill August 2012 - January 2014 \\
            
%\end{resume}
\end{document}







