% LaTeX resume using res.cls
\documentclass[line,mm]{res} 
\usepackage[letterpaper, tmargin=.10in, bmargin=.10in, lmargin = .25in, rmargin=1in, marginparwidth=.01in]{geometry}
\usepackage{hyperref} 
%\usepackage[backend=biber, style=alphabetic, citestyle=authoryear]{biblatex}

\hypersetup{
    colorlinks=true,
    linkcolor=blue,
    urlcolor=blue
}

%\addbibresource{citations.bib}

\begin{document}

% Center the name over the entire width of resume:
 \moveleft.5\hoffset\centerline{\large\bf \href{http://alexrosengarten.com/}{Alexander Sky Rosengarten}}
% address begins here
% Again, the address lines must be centered over entire width of resume:
 \moveleft.5\hoffset\centerline{alex.rosengarten@gmail.com | \href{https://github.com/alxrsngrtn/}{github.com/alxrsngrtn} | (760) 518-7440 | Oakland, CA }

% Draw a horizontal line the whole width of resume:
 \moveleft\hoffset\vbox{\hrule width 7.5in height 1pt}\smallskip
 
% ---------------- EDUCATION ----------------
\section{EDUCATION}  {\sl B.S.} Computer Science \& {\sl B.S.} Cognitive Science \hfill UC San Diego --  La Jolla, CA  \hfill Class of 2016 

% ---------------- TECHNICAL SKILLS ----------------
\section{TECHNICAL SKILLS} 
\begin{tabular}{l@{\hskip 0.25in} l}
\textbf{Languages} & Python, Java, Typescript, C++, JavaScript, C\#, Scala, OCaml, F\# \\
\textbf{Technologies} & Android, Vue.js, Flask, Spring Boot, Reactive Extensions,  SQL/No-SQL, REST, AWS, GCP \\
\textbf{Data Science} & Keras/Tensorflow, Scikit-Learn, OpenCV, PyData Stack, Apache Spark \\
\textbf{Environment} & POSIX, Vim, JetBrains IDEs, Git, Docker, Travis, Jenkins, Jira, Confluence, Windows
\end{tabular}

% ---------------- WORK EXPERIENCE ----------------
% Talk about the challenge that had to be done, the  actions that you took to tackle the challenge, and the results (how things outside of yourself were experienced)		
\section{WORK EXPERIENCE} 
{\bf Software Engineer in Machine Learning } \hfill 2017/11 | Present \\
  Aira Tech Cop -- La Jolla \& San Jose, CA \\
  \textit{ Aira helps blind and low-vision users access visual information via remote assistance with smart glasses. }
  \begin{itemize}  \itemsep -2pt %reduce space between items
    \item Built {\bf core dialog engine (Chloe)} on Android. Integrated STT/TTS system and conversation state management. Built features such as bluetooth pairing, calling an agent, and call rating purely through a voice interface (Java 8). 
    \item Developed cloud-based, \href{https://github.com/aira/object_detector}{{\bf real-time object detector}}.  AI agent used NLP, computer vision, and deep learning for NSF grant, designed to assist blind \& low-vision users (Tensorflow, OpenCV, MQTT).
    \item Technical lead for prototype of {\bf indoor navigation system} (OpenSfM, PCL, Python, Android, ArchGIS). 
    \item Project lead for {\bf Horus}, a batch ETL \& analytics tool for video streams (Python, Docker, AWS).
    \item Creating {\bf image tagging game} to generate dataset integral to company IP (Spring Boot, Vue.js, Typescript).
    \item Created {\bf currency classifier}. Built CNN model. Deployed on Android with Firebase (Tensorflow, Android).
    \item Built {\bf telemetry system}, captures user \& Chloe utterances. Speech simulation key to test automation (AWS IoT).
    \item Making important design \& architecture decisions on AI products, e.g. inter-day call forcasting, OCR reading.
    %\item Added {\bf fisheye correction} in glasses camera driver (OpenCV, Android Native).
    \item Leading agile rituals such as daily standups, sprint plannings and retrospectives.
    \item Founded weekly lecture series to teach engineers machine learning. 
  \end{itemize} 


{\bf Software Engineer} \hfill 2016/08 | 2017/11 \\
  Intuit -- San Diego, CA
  \begin{itemize}  \itemsep -2pt %reduce space between items
    \item Developed features for {\bf ProSeries}, Windows professional tax software (C\#, C++).
    \item {\bf Won business unit hackathon} with novel idea for a telemetry system, presented to senior leadership.
    \item {\bf Led telemetry project}. Took initiative to modernize legacy application with data.
    \item Implemented \href{https://github.com/alxrsngrtn/CrashAnalysisTool}{{\bf application crash analysis tool}}, produced more accurate reports in half the time (Python).
    \item Ideas I designed and executed led to a 50\% reduction in application crashes from prior year. 
  \end{itemize} 

	
{\bf Research Assistant} \hfill            2013/06 | 2015/09 \\
   Natural Computation Laboratory (de Sa Lab), UCSD Cognitive Science Department -- La Jolla, CA
   \begin{itemize}  \itemsep -2pt %reduce space between items
     \item Created \href{https://youtu.be/UMACp0fc9TA}{{\bf in-ear EEG}} device to detect seizures for people with epilepsy (SVMs, OpenBCI).
     \item Built custom neural network layer for \href{https://github.com/alxrsngrtn/LearnedNormPooling}{{\bf learned-norm pooling}} in EEG research application (Theano).
     \item Created hands-on workshops to {\bf teach brain computer interface technology} (OpenBCI, Python).
     \item Developed \href{https://github.com/alxrsngrtn/BrainTag}{{\bf BrainTag}}, a lasertag-based neurofeedback game for children with Autism (Arduino, C). 
     \item Presented research at UCSD Undergraduate Research Conference 2015.
    %\item Wrote data visualization and machine learning toolbox for open source brain-computer interface system (Python, Java, C, Javascript). 
   \end{itemize} 
	
    
% ---------------- PROJECTS ----------------
\section{PROJECTS}
{\bf Research Assistant } \hfill            2018/01 | Present \\
  Vecchio Group, UCSD Nanoengineering Department -- La Jolla, CA
  \begin{itemize}  \itemsep -2pt %reduce space between items   
     \item Published \href{https://arxiv.org/ftp/arxiv/papers/1902/1902.03682.pdf}{{\bf ``Paradigm Shift in Electron-Based Crystallography via Machine Learning''}} (Tensorflow). 
     \item Used OOP design principles for framework to track hyperparameters, ML pipelines in version control (Python).
     \item {\bf Teaching graduate students} programming best-practices and machine learning concepts.  
  \end{itemize} 


{\bf Co-Creator of Computer Music Library} \hfill 	 2017/10 | Present  \\
  Huh, Computer Music (HCM) -- San Diego, CA
  \begin{itemize}  \itemsep -2pt % reduce space between items
    \item Developing \href{https://github.com/alxrsngrtn/huh-computer-music}{{\bf open-source}} Python 3 library for real-time {\bf music generation}.  
    \item Presented {\bf \href{https://youtu.be/bTAFl8P2DkE?t=28m47s}{``Music from Chaos: Audiofying the Lorenz Attractor''}} at PyCon, 2018.
    \item Library utilizes functional reactive programming for live streaming (RxPy).
    \item Tool aims to mimic an analog synthesizer through functional composition (Numpy, Scipy).
  \end{itemize}

% ---------------- INTERESTS ----------------
\section{INTERESTS}
Artificial Intelligence, Biometrics, Functional Programming, Teaching, Accessibility
                 
% ---------------- ACTIVIES & AWARDS ----------------
%\section{ AWARDS \& ACTIVITIES }  
%  {\bf Third Place Winner}, {\it IEEE Brain Computer Interface Hackathon} \hfill September 2016\\
%  {\bf Computer Science Tutor}, {\it UCSD CSE Department} \hfill Spring 2013, Summer 2014 \\
%  {\bf Summer Research Scholar}, {\it Qualcomm Institute/Calit2} \hfill June - August 2013 \\   
%  {\bf President}, {\it Cognitive Science Student Association} \hfill August 2012 - January 2014 \\

\end{document}

