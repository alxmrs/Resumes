% LaTeX resume using res.cls
\documentclass[line,margin]{res} 
%\usepackage{helvetica} % uses helvetica postscript font (download helvetica.sty)
%\usepackage{newcent}   % uses new century schoolbook postscript font 


%\topmargin=-.10in
%\oddsidemargin -.5in
%\evensidemargin -.5in
%\textwidth=6.0in
%\itemsep=0in
%\parsep=0in

\usepackage[letterpaper, tmargin=.10in, bmargin=.10in, lmargin = .25in, rmargin=1.75in, marginparwidth=.01in]{geometry}
\usepackage{hyperref} 


\begin{document}

% Center the name over the entire width of resume:
 \moveleft.5\hoffset\centerline{\large\bf Alexander Sky Rosengarten}
% address begins here
% Again, the address lines must be centered over entire width of resume:
 \moveleft.5\hoffset\centerline{alex.rosengarten@gmail.com | \href{https://github.com/alxrsngrtn/}{github.com/alxrsngrtn} | (760) 518-7440 | San Diego, CA }

% Draw a horizontal line the whole width of resume:
 \moveleft\hoffset\vbox{\hrule width 7.5in height 1pt}\smallskip
 
%\section{OBJECTIVE}
%Looking to make an meaningful impact as a machine learning engineer or full-stack web developer. 
 
\section{EDUCATION} University of California, San Diego, {\it La Jolla, CA} \hfill {\sl B.S.} Computer Science \& {\sl B.S.} Cognitive Science
		  

\section{TECHNICAL \\ SKILLS} 
	
%\begin{tabular}{l l l}
% 								Experience & Languages & Technologies \\ \hline
%								Adept	&   \parbox[t]{2.4in}{Python, Java, Javascript, C/C++,  Matlab/Julia}   & %\parbox[t]{2.4in}{Numpy/Pandas, Angular.js, Webpack, Git, Theano, Jet Brains IDEs} \\
%								Proficient &  \parbox[t]{2.4in}{C\#, Scala} & \parbox[t]{2.4in}{Docker, Express.js, Ionic/%Cordova, SQL, Apache Spark, Spring MVC, }\\ % Hadoop
%								% Adobe Creative Suite, Android Studio, Arduino, Eclipse, GDB, Git, Unix, Vim
%\end{tabular}


%Languages & Python, C\#, C++, JavaScript, Java, Matlab/Octave/Julia, F\#, Scala  \\
%Frontend & Angular.js, Jasmine, Ionic/Cordova, Grunt/Gulp, Webpack \\
%Backend & SQL, Cassandra, Flask, Node.js, Django, Neo4j, Spring boot \\
%Data & Keras/Tenserflow/Theano, Scikit-Learn, Apache Spark \\
%Environment & Git, Docker, Vagrant, Unix/Windows, SVM, Perforce

\begin{tabular}{l@{\hskip 0.25in} l}
\textbf{Languages} & Python 3, Java 1.8, C++, C\#, JavaScript, Matlab, Scala/F\# \\
\textbf{Technologies} & Android, Reactive Extensions, SQL/No-SQL, AMQP, REST, AWS, Google Cloud \\
\textbf{Data Science} & Keras/Tenserflow, Scikit-Learn, OpenCV, Numpy, Pandas, Jupyter, Apache Spark \\
\textbf{Environment} & POSIX, Vim, JetBrains IDEs, Git, Docker, Vagrant, Windows
\end{tabular}

% Talk about the challenge that had to be done, the  actions that you took to tackle the challenge, and the results (how things outside of yourself were experienced)		
\section{WORK EXPERIENCE} 
{\bf Machine Learning/Fullstack Engineer } \hfill            December 2017 - Present \\
                Aira, La Jolla, CA
                 \begin{itemize}  \itemsep -2pt %reduce space between items
		\item Developing {\bf Chloe}, a mobile-AI for new product line (Android, RxJava). 
		\item Building {\bf core dialog engine}. Includes a STT/TTS system and conversation state management. Used to build features such as bluetooth pairing, calling an agent, call rating purely throuch a voice interface. 
		 \item Built cloud-based, \href{https://github.com/aira/object_detector}{{\bf real-time object detector}}.  AI agent used NLP, computer vision, and deep learning for NSF grant, designed to assist blind/low-vision users. (Tensorflow, OpenCV, MQTT).
		\item AI agent can describe the color, amount, and type of 100 classes of objects in under half a second.
                \item Created {\bf wake word detector}. Built a CNN model and audio processing pipeline (Tensorflow).
                \item Built a {\bf telemtry system} that captures speech utterances from users and the device. Speech simulation added for automated testing (AWS IoT, MQTT, DynamoDB).
		\item Started a weekly lecture series to teach engineers machine learning. 
                 \end{itemize} 


{\bf Research Assistant } \hfill            January 2018 - Present \\
                UCSD Nanoengineering Department, La Jolla, CA
                 \begin{itemize}  \itemsep -2pt %reduce space between items
                \item Building {\bf crystal structure classifier}. Built a data pipeline and model to infer crystal structure types based on images of defraction patterns, achieved 94\% accuracy accross four classes (Keras/Tensorflow).
                \item {\bf Teaching graduate students} programming best practices and machine learning concepts.  
                 \end{itemize} 

		{\bf Software Engineer} \hfill            August 2016 - November 2017 \\
                Intuit, San Diego, CA
                 \begin{itemize}  \itemsep -2pt %reduce space between items
                 \item Developed C\# and C++ features for {\bf ProSeries}, Windows professional tax software. 
                 \item {\bf Won business unit hackathon} with novel idea for a telemetry system, presented to senior leadership.
                 \item {\bf Lead telemetry project}, which was an outstanding need for over three years. Telemetry system is critical for new design and refactoring initiatives as well as making important business decisons.
                 \item Wrote \href{https://github.com/alxrsngrtn/CrashAnalysisTool}{{\bf application crash analysis tool}} that produces more accurate reports in half the time (Python).
                 \end{itemize} 



%{\bf Content Developer } \hfill            February - April 2016 \\
%                AI-Master, San Diego, CA
%                 \begin{itemize}  \itemsep -2pt %reduce space between items
%                \item Wrote interactive tutorials in Python at Machine Learning education start-up.
%                \item Developed lesson plans on Support Vector Machines in Jupyter/IPython Notebooks.
%                \item Established version control system (Github) for content team using open source pull-request model.
%                 \end{itemize} 
%

		{\bf Software Engineering Intern} \hfill            March - August 2016 \\
                Ingenu, San Diego, CA
                 \begin{itemize}  \itemsep -2pt %reduce space between items
                 \item Developed {\bf Supernova}, a single-page web application to manage IoT devices (Angular 1).
                 \item�Reduced application load time by 44\% and KB size by 40\% via minification, lazy loading, and CDN.
                 \item Prototyped, designed, and developed {\bf RSS feed widget} (Spring boot, Angular).
                 \end{itemize} 
	
		{\bf Research Assistant} \hfill            June 2013 - September 2015 \\
                Natural Computation Laboratory, UCSD Dept. of Cognitive Science, La Jolla, CA
                 \begin{itemize}  \itemsep -2pt %reduce space between items
                 \item Created \href{https://youtu.be/UMACp0fc9TA}{{\bf in-ear EEG}} device to detect seizures for people with epilepsy (SVMs, OpenBCI).
                 \item Build custom neural network layer for \href{https://github.com/alxrsngrtn/LearnedNormPooling}{{\bf learned-norm pooling}} in EEG research application (Theano).
                 \item Created hands-on workshops to {\bf teach brain computer interface technology} (OpenBCI, Python).
                 \item Developed \href{https://github.com/alxrsngrtn/BrainTag}{{\bf BrainTag}}: a lasertag-based neurofeedback game for children with Autism (Arduino, C). 
                 %\item Wrote data visualization and machine learning toolbox for open source brain-computer interface system (Python, Java, C, Javascript). 
                 \end{itemize} 
	
		%{\bf Computer Science Tutor} \hfill 	April - June 2013, August 2014 \\
		%UCSD Computer Science Department, La Jolla, CA
                 %\begin{itemize}  \itemsep -2pt % reduce space between 
                 %\item Spring 2013, Matlab Programming: Tutored students, debugged code, graded exams and assignments.
                 %\item Summer 2014, Systems Programing: Tutored students in C and ARM Assembly programming course, implemented on Raspberry Pi. Found errors and debugged code (ARM, C, Raspberry Pi). 
                 %\end{itemize}
                 
    
\section{PROJECTS}
                 {\bf Huh, Computer Music (HCM) } \hfill 	 October 2017 - Present 
                 \begin{itemize}  \itemsep -2pt % reduce space between items
                 \item Developing \href{https://github.com/alxrsngrtn/huh-computer-music}{open-source} Python 3 library for realtime music generation.  
		 \item Presented {\bf ``\href{https://youtu.be/bTAFl8P2DkE?t=28m47s}{Music from Chaos: Audiofying the Lorenz Attractor}''} at PyCon, 2018.
                 \item Library utilizes functional reactive programming for live streaming (RxPy).
                 \item Tool aims to mimic an analog synthesizer through functional composition (Numpy, Scipy).
                 \end{itemize}

                 % include quantitative results
%                {\bf Forward 135 - Full Stack Developer} \hfill 	 April - June 2016 \\
%                Server-Side Web Applications Course, La Jolla, CA 
%                 \begin{itemize}  \itemsep -2pt % reduce space between items
%                 \item Built a full-stack architecture for efficient change propagation from server to client (JS ES6, Webpack)
%                 \item Implemented MVVM solution that sends only diffs from database to client via WebSockets (Node.js). 
%                 \item Performed Incremental View Maintenance via materialized views and database triggers (PostgreSQL). 
%                 \end{itemize}
                 
% {\bf In-Ear EEG - Software Team Member} \hfill 	 April - June 2015 \\
%                Embedded Systems Project Course, La Jolla, CA 
%                 \begin{itemize}  \itemsep -2pt % reduce space between items
%                 \item Worked in team to create In-Ear EEG device for people with epilepsy.
%                 \item Used ensemble of SVMs to detect presence of epileptic seizures in realtime data (Matlab, OpenBCI).
%                 \item Prototyped server that stored 20 minutes of brainwave data once seizures are detected (Python, SQL). 
%                 \end{itemize}
                
%\section{COURSEWORK}
   
%   \begin{tabular}{l l l}
%   $\bullet$ Advanced Data Structures  & $\bullet$ Advanced Machine Learning & $\bullet$ Algorithms \\
%   $\bullet$ Compiler Construction & $\bullet$ Computer Vision & $\bullet$ Neural Networks \\ 
%   $\bullet$ Operating Systems  &  $\bullet$ Programming Languages & $\bullet$ Server-Side Web Applications  \\
%    \end{tabular}
                 
\section{ AWARDS \& ACTIVITIES }  
	    {\bf Third Place Winner}, {\it IEEE Brain Computer Interface Hackathon} \hfill September 2016\\
	    {\bf Trainee}, {\it Temporal Dynamics of Learning Center} \hfill January 2014 - June 2015 \\  
	    {\bf Computer Science Tutor}, {\it UCSD CSE Department} \hfill Spring 2013, Summer 2014 \\
	    {\bf Summer Research Scholar}, {\it Qualcomm Institute/Calit2} \hfill June - August 2013 \\   
            {\bf President}, {\it Cognitive Science Student Association} \hfill August 2012 - January 2014 \\
            
%\end{resume}
\end{document}







